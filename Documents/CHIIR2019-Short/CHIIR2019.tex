%% LyX 2.2.1 created this file.  For more info, see http://www.lyx.org/.
%% Do not edit unless you really know what you are doing.
\documentclass[sigconf,anonymous=true,authordraft=false]{acmart}
\usepackage[latin9]{inputenc}
\usepackage{array}
\usepackage{enumitem}
\usepackage{graphicx}

\makeatletter

%%%%%%%%%%%%%%%%%%%%%%%%%%%%%% LyX specific LaTeX commands.
%% Because html converters don't know tabularnewline
\providecommand{\tabularnewline}{\\}
%% A simple dot to overcome graphicx limitations
\newcommand{\lyxdot}{.}


%%%%%%%%%%%%%%%%%%%%%%%%%%%%%% Textclass specific LaTeX commands.
\newlength{\lyxlabelwidth}      % auxiliary length 

%%%%%%%%%%%%%%%%%%%%%%%%%%%%%% User specified LaTeX commands.
%\renewcommand\footnotetextcopyrightpermission[1]{} % removes footnote with conference information in first column
%\pagestyle{plain} % removes running headers


%\makeatletter
%\renewcommand\@formatdoi[1]{\ignorespaces}
%\makeatother

\usepackage{amssymb}
\usepackage{url}
\usepackage{enumitem}
\usepackage{amstext}
\usepackage{graphicx}
\usepackage{booktabs} % For formal tables


\usepackage{subfiles}

\usepackage{subfigure}
\usepackage[ruled,vlined,linesnumbered]{algorithm2e}
\DeclareMathOperator*{\argmax}{arg\,max}

\newcommand\varlist{,\makebox[0.8em][c]{.\hfil.\hfil.},} 

% Copyright
%\setcopyright{none}
%\setcopyright{acmcopyright}
%\setcopyright{acmlicensed}
\setcopyright{rightsretained}
%\setcopyright{usgov}
%\setcopyright{usgovmixed}
%\setcopyright{cagov}
%\setcopyright{cagovmixed}


% DOI
\acmDOI{10.475/123_4}

% ISBN
\acmISBN{123-4567-24-567/08/06}

%Conference
\acmConference[WOODSTOCK'97]{ACM Woodstock conference}{July 1997}{El
  Paso, Texas USA} 
\acmYear{1997}
\copyrightyear{2016}

\acmPrice{15.00}

\fancyhead{}
\settopmatter{printacmref=false, printfolios=false}

\makeatother

\begin{document}
%\title{An Information Retrieval Perspective of Filter Selection for Adaptive User Interfaces } 
%\title{An F1-Score Optimization Based Method for Clustering Geo-Temporal Data} 
%\title{Clustering Geo-Temporal Data Using Greedy Based Algorithms By Optimizing F1-Score}
%\title{Clustering Geo-Temporal Data Using Greedy Algorithms Based on F1-Score Optimization} 
%\title{Optimizing F1-Score Using Greedy Algorithms for Clustering Geo-Temporal Data} 

% Also we take a supervised perspective vs. previously unsupervised approaches
%\title{Relevance-driven F1-Score Optimization\\ for Filtering in Visual Information Displays}
% Full CHIIR2018
\title{Relevance-driven Clustering for Visual Information Retrieval}
% Short CHIIR2018
%\title{Toward Relevance-driven Clustering for Visualizing Geo-temporal Search Results}
% 
%\title{Relevance-driven Clustering for Interactive Visual Search}

\author{Mohamed Reda Bouadjenek}
%\orcid{1234-5678-9012}
\affiliation{%
  \institution{The University of Toronto}
  \streetaddress{Department of Mechanical and\\ Industrial Engineering}
  \city{Toronto} 
  \state{Ontario} 
   \postcode{M5S 3G8}
  \country{Canada}
}
\email{mrb@mie.utoronto.ca}

\author{Scott Sanner}
\affiliation{%
  \institution{The University of Toronto}
  \streetaddress{Department of Mechanical and\\ Industrial Engineering}
  \city{Toronto} 
  \state{Ontario} 
   \postcode{M5S 3G8}
  \country{Canada}
}
\email{ssanner@mie.utoronto.ca }


\author{Yihao Du}
\affiliation{%
  \institution{The University of Toronto}
 \streetaddress{Department of Mechanical and\\ Industrial Engineering}
  \city{Toronto} 
  \state{Ontario} 
   \postcode{M5S 3G8}
  \country{Canada}
 }
\email{duyihao@mie.utoronto.ca }





\newcommand{\subfour}[1]{\vspace*{3mm}{\noindent\bf #1}}  
\newcommand{\subsubfour}[1]{\vspace*{1mm}{\noindent\bf #1}} 
\newtheorem{problem}{\textbf{Problem}}

% VSI -- is it really "search"?
% VID
% Information Visualization
% AUI???
\begin{abstract}
Many existing works have used unsupervised clustering methods such as K-means to aggregate results and reduce clutter in visual information retrieval.  These ideas build on the ``cluster hypothesis'' of information retrieval, which posits that documents in the same cluster are likely to address similar relevance needs for the search task.  Unfortunately, the use of unsupervised methods does not necessarily guarantee that the clusters themselves are relevant.  In this paper, we address this deficiency via a novel \emph{relevance-driven} clustering objective and optimization method intended for visual information retrieval.  We develop a greedy algorithm for efficient optimization and benchmark it against (a more expensive) optimal solution obtained through a Mixed Integer Linear Programming (MILP) formulation.  Finally, we undertake a user study with 24 subjects to evaluate whether this new relevance-driven clustering method improves human performance on visual search tasks and demonstrate that it results in higher search task recall and accuracy vs. time in comparison to K-means and a non-aggregation baseline. 

\noindent
{\bf Keywords:} Visual Search Interfaces; Relevance-driven Clustering.%; Constrained Optimization.

\end{abstract}




%\keywords{Adaptive User Interfaces; Search Algorithms;  mathematical optimization for IR.}
\maketitle



\section{Introduction}
\subfile{Introduction}


\section{Framework and background}
\label{sec:Framework}
\subfile{Framework}


\section{Relevance-driven clustering}
\label{sec:Algorithms}
\subfile{Algorithms}

%\subsection{Optimization search}
%\subfile{Optimal}

\section{Experimental setup}


\label{sec:setup}
\subfile{Setup}


\section{Off-line evaluation}
\label{sec:OfflineEval}
\subfile{OfflineEval}

\section{User study}
\label{sec:UserStudy}
\subfile{UserStudy}

%\section{Related work}
%\label{sec:RelatedWork}
%\subfile{RelatedWork}

\section{Conclusions and Future work}
\label{sec:Conclusions}
\subfile{Conclusions}


%\bibliographystyle{abbrv}
\bibliographystyle{unsrt}
\bibliography{biblio}
%\appendix
%\subfile{Appendix}



\end{document}
